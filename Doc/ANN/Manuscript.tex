\documentclass{article}

% ===================== preamble =========================

\usepackage[dvips]{graphicx}

\usepackage{xcolor}
% the deleting line is imported by package ulem
\usepackage{ulem}

\usepackage{amssymb}
\usepackage{amsmath}

% new mathematical symbol
% \newcommand{\dd}{\; \mathrm{d}}

% appendix
\usepackage[title]{appendix}

% for nomenclature
\usepackage{ifthen}
\usepackage{framed}
\usepackage{multicol}
\usepackage[noprefix]{nomencl}
\makenomenclature

% grouped nomenclature
\renewcommand{\nomgroup}[1]{%
	\ifthenelse{\equal{#1}{L}}{%
		\medskip\item[\large\textbf{\textsf{Latin}}]\medskip%
	}{%
		\ifthenelse{\equal{#1}{G}}{%
			\medskip\item[\large\textbf{\textsf{Greek}}]\medskip%
		}{%
			\ifthenelse{\equal{#1}{N}}{%
				\medskip\item[\large\textbf{\textsf{Dimensionless}}]\medskip%
			}{%
				\ifthenelse{\equal{#1}{B}}{%
					\medskip\item[\large\textbf{\textsf{Matrix and vector}}]\medskip%
				}{}
			}
		}
	}
}

% set the line spacing of nomenclature
\setlength{\nomitemsep}{-\parsep}

\renewcommand*\nompreamble{\begin{multicols}{2}}
\renewcommand*\nompostamble{\end{multicols}}

\newcommand{\revised}[1]{\textcolor{blue}{#1}}
\newcommand{\deleted}[1]{\textcolor{red}{\sout{#1}}}

\newcommand{\dd}{\: \mathrm{d}}
\newcommand{\parpar}[2]{\frac{\partial {#1}}{\partial {#2}}}
\newcommand{\parpard}[2]{ \frac{\partial^2 {#1}}{\partial {#2}^2} }

\newcommand{\inbrace}[1]{\left\{ #1 \right\}}
\newcommand{\inbracesmall}[1]{\{ #1 \}}
\newcommand{\inbracket}[1]{\left[ #1 \right]}

\newcommand{\myeqref}[1]{Eq.~\eqref{#1}}
\newcommand{\myfigref}[1]{Fig.~\ref{#1}}
\newcommand{\mytabref}[1]{Table.~\ref{#1}}

\title{The Title}

% first author, myself
\author{Y.Y.~Hu}

% =================== end of preamble ====================

% ===================== document =========================
%
% Created: Jun. 19th, 2017 @ SJTU
%
% ========================================================

\begin{document}

% ==============================
% == main body of the article ==
% ==============================

\maketitle

\section{Introduction}

The Introduction. 

\section{The second section title}
\subsection{Subsection}
Fig.~\ref{fig:Coordinates}, Eq.~\eqref{eq:OriginalEnery}.
\begin{align} \label{eq:OriginalEnery}
	& \rho c \left( \frac{\partial T}{\partial t} + u\frac{\partial T}{\partial x} + v\frac{\partial T}{\partial y} + w\frac{\partial T}{\partial z} \right) = k_f \frac{\partial^2 T}{\partial z^2} \nonumber \\
	&+ \eta\left( \left(\frac{\partial u}{\partial z}\right)^2 + \left( \frac{\partial v}{\partial z} \right)^2 \right)
\end{align}
Where $c$, $k_f$ and $\eta$ are the specific heat, thermal conductivity and dynamic viscosity of the lubricant, respectively. Eq.~\eqref{eq:OriginalEnery} describes the temperature distribution. 

% == acknowledgment ==
\section*{Acknowledgements}
The acknowledgements.

% == bibliography ==
%\section*{References}
\bibliographystyle{model3-num-names}
\bibliography{TheBBLFile}

% == appendix ==
\numberwithin{equation}{section}
\begin{appendices}

\section{Some symbols used in this article}

\end{appendices}


% ==================== figures ================================

% \begin{figure}[htb]
% 	\centering
% 	\includegraphics[width=0.95\linewidth]{FIG01}
% 	\caption{Coordinate system}
% 	\label{fig:Coordinates}
% \end{figure}

% ====================== tables ===================

\begin{table}[!hbtp]
	\caption[Table of parameters]{The parameters of the bearing model}
	\begin{center}
	\begin{tabular}{l l}
		\hline
		\textbf{Description} & \textbf{Quantity} \\
		\hline
		\textbf{Geometry} & \\
		Outer radius, $r_2$ (m) & 0.45 \\
		\\
		\textbf{Operation Conditions} & \\
		Rotational speed (rpm) & 1780 \\
		\hline
	\end{tabular}
	\end{center}
\end{table}

\end{document}
